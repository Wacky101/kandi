% ----------------- joitakin makroja ----------------------------------
%
% \newcommand{\sinunKomentosi}[argumenttienMäärä]{komennot%
% voiJakaaRiveille%
% jaArgumenttienViittaus#1,#2,#argumenttienMäärä}
% Joskus voi olla tarpeen kommentoida jotakin. Ei suositella. 
% Äläkä unohda lopulliseen! 
\newcommand{\Kommentti}[1]{\fcolorbox{black}{red}{OMA KOMMENTTI: #1}}
%\newcommand{\Kommentti}[1]{\fbox{\textbf{OMA KOMMENTTI:} #1}}
% Käyttö: Kilometri on 1024 metriä. \Kommentti{varmista tämä vielä}.
% Eli newcommand:n komentosanan jälkeen hakasaluissa argumenttien lkm,
%  ja argumentteihin viitataa #1, #2, ...

%  Comment out this \DRAFT macro if this version no longer is one!  XXX
%\newcommand{\DRAFT}{\begin{center} {\it DRAFT! \hfill --- \hfill DRAFT!
%\hfill --- \hfill DRAFT! \hfill --- \hfill DRAFT!}\end{center}}

%  Use this \DRAFT macro in the final version - or comment out the 
%  draft-command
% \newcommand{\DRAFT}{~}

% %%%%%%%% MATEMATIIKKA %%%%%%%%%%%%%%%%%

% Määrätty integraali
\newcommand{\myInt}[4]{%
\int_{#1}^{#2} #3 \, \textrm{d}{#4}}

% http://kapsi.fi/jks/satfaq/
%\newcommand{\vii}{\mathop{\Big/}}
%\newcommand{\viiva}[2]{\vii\limits_{\!\!\!\!{#1}}^{\>\,{#2}}}
%%\[ \intop_0^{10} \frac{x}{x^2+1} \,\mathrm{d}x
%%= \viiva{0}{10} \frac{1}{2}\ln(x^2+1) \]

% matht.sty, Simo K. Kivelä, 01.01.2002, 07.04.2004, 19.11.2004, 21.02.2005
% Kokoelma matemaattisten lausekkeiden kirjoittamista helpottavia
% määrittelyjä.

% 07.04.2004 Muutama lisäys ja muutos tehty: \ii, \ee, \dd, \der,
% \norm, \abs, \tr.
%
% 19.11.2004 Korjattu määrittelyjä: \re, \im, \norm;
% lisätty \trp (transponointi), \hrm (hermitointi), \itgr (rakenteellinen
% integraali), ympäristö Cmatrix (hakasulkumatriisi);
% vanha transponointi \tr on mukana edelleen, mutta ei suositella.

% Pakotettu rivinvaihto, joka voidaan tarvittaessa määritellä
% uudelleen: 

%\newcommand{\nl}{\newline}

% Logiikan symboleja (<=> ja =>) hieman muunnettuina:

%\newcommand{\ifftmp}{\;\Leftrightarrow\;}
%\newcommand{\impltmp}{\DOTSB\;\Rightarrow\;}

% 'siten, että' -lyhenne ja hattupääyhtäläisyysmerkki vastaavuuden
% osoittamiseen: 

%\newcommand{\se}{\quad \text{siten, että} \quad}
%\newcommand{\vs}{\ {\widehat =}\ }

% Lukujoukkosymbolit:

%\newcommand{\N}{\ensuremath{\mathbb N}}
%\newcommand{\Z}{\ensuremath{\mathbb Z}}
%\newcommand{\Q}{\ensuremath{\mathbb Q}}
%\newcommand{\R}{\ensuremath{\mathbb R}}
%\newcommand{\C}{\ensuremath{\mathbb C}}

% Reaali- ja imaginaariosa, imaginaariyksikkö:

%\newcommand{\re}{\operatorname{Re}}
%\newcommand{\im}{\operatorname{Im}}
%\newcommand{\ii}{\mathrm{i}}

% Differentiaalin d, Neperin luku:

%\newcommand{\dd}{\mathrm{d}}
%\newcommand{\ee}{\mathrm{e}}

% Vektorimerkintä, joka voidaan tarvittaessa määritellä uudelleen
% (tämä tekee vektorit lihavoituina):

%\newcommand{\V}[1]{{\mathbf #1}}

% Kulmasymboli:

%\renewcommand{\angle}{\sphericalangle}

% Vektorimerkintä, jossa päälle pannaan iso nuoli;
% esimerkiksi \overrightarrow{AB} (tämmöisiä olemassaolevien
% symbolien uudelleenmäärittelyjä ei kyllä pitäisi tehdä):

%\renewcommand{\vec}[1]{\overrightarrow{#1}}

% Vektoreiden vastakkaissuuntaisuus:

%\newcommand{\updownarrows}{\uparrow\negthinspace\downarrow}

% Itseisarvot ja normi:

%\newcommand{\abs}[1]{{\left\vert#1\right\vert}}
%\newcommand{\norm}[1]{{\left\Vert #1 \right\Vert}}

% Transponointi ja hermitointi:

%\newcommand{\trp}[1]{{#1}\sp{\operatorname{T}}}
%\newcommand{\hrm}[1]{{#1}\sp{\operatorname{H}}}

% Vanha transponointi; jäljellä yhteensopivuussyistä, ei syytä käyttää.
%\newcommand{\tr}{{}^{\text T}}

% Arcus- ja area-funktiot, jossa päähaara osoitetaan nimen päälle
% vedetyllä vaakasuoralla viivalla (alkaa olla vanhentunutta,
% voitaisiin luopua):

%\newcommand{\arccot}{\operatorname{arccot}}
%\newcommand{\asin}{\operatorname{\overline{arc}sin}}
%\newcommand{\acos}{\operatorname{\overline{arc}cos}}
%\newcommand{\atan}{\operatorname{\overline{arc}tan}}
%\newcommand{\acot}{\operatorname{\overline{arc}cot}}

%\newcommand{\arsinh}{\operatorname{arsinh}}
%\newcommand{\arcosh}{\operatorname{arcosh}}
%\newcommand{\artanh}{\operatorname{artanh}}
%\newcommand{\arcoth}{\operatorname{arcoth}}
%\newcommand{\acosh}{\operatorname{\overline{ar}cosh}}

% Signum, syt, pyj:

%\newcommand{\sg}{\operatorname{sgn}}
%\renewcommand{\gcd}{\operatorname{syt}}
%\newcommand{\lcm}{\operatorname{pyj}}

% Lyhennemerkintöjä: derivaatta, osittaisderivaatta, gradientti,
% derivaattaoperaattori, vektorin komponentti, integraalin ylä- ja
% alasumma, Suomessa (ja Saksassa?) käytetty integraalin sijoitus-
% merkintä, integraali (rakenteellinen määrittely):

%\newcommand{\der}[2]{\frac{\dd #1}{\dd #2}}
%\newcommand{\osder}[2]{\frac{\partial #1}{\partial #2}}
%\newcommand{\grad}{\operatorname{grad}}
%\newcommand{\Df}{\operatorname{D}} 
%\newcommand{\comp}{\operatorname{comp}}
%\newcommand{\ys}[1]{\overline S_{#1}}
%\newcommand{\as}[1]{\underline S_{#1}}
%\newcommand{\sijoitus}[2]{\biggl/_{\null\hskip-6pt #1}^{\null\hskip2pt #2}} 
%\newcommand{\itgr}[4]{\int_{#1}^{#2}#3\,\dd #4}

% Matriiseja, joille voidaan antaa alkioiden sijoittamista sarakkeen
% vasempaan tai oikeaan reunaan tai keskelle osoittava lisäparametri
% (l, r tai c); ympärillä kaarisulut, hakasulut, pystyviivat (determinantti)
% tai ei mitään;
% esimerkiksi \begin{cmatrix}{ll}1 & -1 \\ -1 & 1 \end{cmatrix}:

%\newenvironment{cmatrix}[1]{\left(\begin{array}{#1}}{\end{array}\right)}
%\newenvironment{Cmatrix}[1]{\left[\begin{array}{#1}}{\end{array}\right]}
%\newenvironment{dmatrix}[1]{\left|\begin{array}{#1}}{\end{array}\right|}
%\newenvironment{ematrix}[1]{\begin{array}{#1}}{\end{array}}

% Kaunokirjoitussymboli:

%\newcommand{\Cal}{\mathcal}

% Isokokoinen summa:

%\newcommand{\dsum}[2]{{\displaystyle \sum_{#1}^{#2}}}

% Tuplaintegraali umpinaisen pinnan yli; korvataan jos parempi löytyy:
%\newcommand\oiint{\begingroup
% \displaystyle \unitlength 1pt
% \int\mkern-7.2mu
% \begin{picture}(0,3)
%   \put(0,3){\oval(10,8)}
% \end{picture}
% \mkern-7mu\int\endgroup}
