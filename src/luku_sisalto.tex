% --------------------------------------------------------------------

\section{Inledning}
Internet of Things (IoT) är ett begrepp som fått stor synlighet inom både
industrin och konsumentmarknaden. Företag strävar efter att digitalisera sin
verksamhet med hjälp av IoT. Detta ger företagen tillgång till värdefull
realtidsinformation om sina tjänster, produkter, infrastrukturer och kunder.
Förtagen kan utnyttja realtidsinformationen till att minska 
upprätthållningskostnader genom att identifiera maskinfel förrän de hinner
uppstå. IoT-system automatiserar arbete genom att låta maskiner kommunicera
sinnsemellan och minskar därmed behovet för männsklig interaktion.
På konsumentmarknaden har IoT-lösningar som bl.a. aktivitetsarmband, smarta hus
och smart TV:n uppståt. 

Då IoT-system växer i storlek och ökar i komplexitet, ökar även 
upprätthållningskostnader och risker för systemfel.
För att hantera komplexiteten i stora IoT-system, kan företag använda
IoT-plattformar. IoT-plattformar effektiverar utveckligen av IoT-lösningar
och gör det lättare för utvecklare att hantera ett ökande antal uppkopplade
maskiner.

%#GAMMALT BEGIN
%Internet of Things (IoT) är ett av de mest aktuella ämnen i företagsverksamhet
%och forskning. I industrin förses existerande maskiner med sensorer, som
%skickar och mottar information över Internet. Fel kan upptäckas proaktivt
%genom att analysera dataflödet från apparaterna och därmed minska
%underhållskostnader.
%På konsumentmarknaden utvecklas kroppsnära teknik, som samverkar med
%mobiltelefoner och synkroniseras med molntjänster.

%Ju större och mera komplexa IoT enablerade system blir, desto mera tid och
%kapital krävs det för att utveckla och upprätthålla dem.
%För att simplifiera och effektivera utvecklingen och underhållningen av
%IoT system, har IoT plattformar utvecklats. Plattformarna kan erbjudas
%som kompletta tjänster eller som programmeringsverktyg i form av
%integrerade utvecklingsmiljöer (IDE).
%#GAMMALT END


\subsection{Problembeskrivning}
För att företag skall lyckas lansera IoT-system med hjälp av IoT-plattformar,
kräver utvecklarna information om kapaciteten och egenskaperna hos 
plattformarna.
Valet av IoT-plattform inverkar på IoT-systemets framgång på samma sätt
som valet av teknologi påverkar traditionella IT-system.
Det ökande antalet tillgängliga IoT-plattformar för med sig många
valmöjligheter, men försvårar samtidigt valet av rätt plattform för rätt
ändamål. Kompilerad information om egenskaper hos tillgängliga IoT-plattformar 
underlättar valet av plattform.


\subsection{Syfte}
Syftet med detta arbete är att identifiera allmänna egenskaper och funktioner
hos IoT-plattformar, samt undersöka hur tillgängliga IoT-plattformar stöder
dessa funktioner.

\subsection{Avgränsningar}
Detta arbete behandlar endast egenskaper och funktioner hos IoT-plattformar.
Arbetet tar inte ställning till plattformarnas ställning på marknaden och
behandlar inte plattformarnas tekniska arkitektur på detaljnivå.

\subsection{Material och metoder}
Forskningen utfördes som en explorativ, kvalitativ undersökning och
Industrial Internet Consortiums (IIC) nätsida fungerade som startpunkt för den.
IIC är ett konsortium, bildat av bl.a. Intel, IBM och AT\&T, som jobbar
för att utveckla IoT-standarder. Över 100 företag och universitet är medlemmar 
i IIC och dessa delar på ansvaret för att utveckla globala IoT lösningar.
Genom att granska medlemmarnas nätsidor, identifierades IoT plattformar.
Ifall en medlem samarbetade med ett eller flera företag, som också länkades
till IoT, kunde forskningen utvidgas till dessa företag.
Information om plattformarnas egenskaper och funktioner hittades genom
utvecklarnas nätsidor, vitböcker och plattformarnas dokumentation.

De egentliga egenskaperna hos plattformarna kan avvika från resultaten
som presenteras i detta arbete. Detta beror på att resultaten bygger på
information given av plattformutvecklarna själva, och kräver verifiering
av en utomstående part.

\pagebreak



% --------------------------------------------------------------------

\section{Internet of Things-plattform}
\label{sec:esimluku}
Detta kapitel 

\subsection{Teknisk plattform}

En teknisk plattform är en kombination av hårdvara och mjukvara, som
tillsammans bildar en miljö, där applikationer kan utvecklas och köras i.
Plattformar underlättar utvecklingen av applikationer, eftersom utvecklaren
kan fokusera på själva lösningen istället och delegera infrastrukturen
till plattformen.
Detta gäller även i fallet av IoT plattformar. Utvecklaren kan allokera
största delen av sina resurser till att skapa värde för slutanvändaren,
medan plattformen ansvarar för de funktioner som stöder utvecklingen.

\subsection{Typiska drag för Internet of Things-plattformar}

\subsubsection{Skalbarhet}
Skalbarhet är ett mått på hur ett datasystem hanterar ett ökande antal 
uppkopplade maskiner \cite{scalability_def}. Ett horisontellt skalbart system
kan öka det maximala antalet maskiner som kan kopplas upp genom att öka
antalet fysiska servrar.
Exempelvis kan ett horisontellt skalbart system hantera 1000 uppkopplade
enheter med 1 server. För att hantera 2000 uppkopplade enheter, måste systemet
öka antalet servrar till 2. Förhållandet mellan antalet uppkopplade enheter
och antalet servrar förblir konstant, då antalet uppkopplade enheter ökar.

IoT-system måste klara av att hantera ett växande antal uppkopplade klienter
och ett ökande informationsflöde på ett stabilt sätt. För att systemet som
helhet skall vara skalbart, måste även den underliggande plattformen vara det.

\subsubsection{Sammanslagning av data}
Information förmedlas i form av alarm och händelser mellan IoT-enablerade
maskiner och det centrala systemet. I systemet är dessa alarm och händelser
bundna till åtgärder som utförs då informationen når systemet. Åtgärderna
kan utföra enkla uppgifter, som att uppdatera klientens fysiska läge
i systemet. De kan också vara en del i en komplex kedja, där flera åtgärder
utförs i en specifik ordning.

IoT-plattformarna kan erbjuda sk. Mashup-funktioner, som skapar och hanterar
regler för hurdana åtgärder skall utföras till följd av alarm och händelser.
\cite{mashup_def} 

\subsubsection{Säkerhet}
För att definiera säkerhet i informationssystem, kan CIA-modellen användas.
Ett säkert informationssystem tar ställning till datans 
\Kommentti{konfidentialitet, integritet och tillgänglighet}. Konfidentialitet
hänvisar till regler för vilka användare eller grupper har rättigheter
att läsa och modifiera informationen. Integritet hänvisar till att innehållet
av informationen inte förändras mellan faserna då en berättigad användare skapar
informationen och då en berättigad användare läser informationen.
Tillgängligheten av information hänvisar till att informationen är nåbar.
\Kommentti{Sauce missing}

Ett IoT-system med ett stort antal uppkopplade klienter är komplexiteten
för att hantera datasekretess hög. IoT-plattformar tar ställning till detta
genom att förse utvecklaren med verktyg för att hantera användarrättigheter
mellan klienterna. Plattformarna kan erbjuda säkra överföringskanaler
med hjälp av SSL/TLS-protokoll \Kommentti{Sauce för protokollen} för att 
förhindra avlyssning eller modifiering av dataflödet mellan klienterna och det
centrala systemet.

\subsubsection{Kommunikationsprotokoll}
Data kan överföras från maskin till maskin med hjälp av flera olika protokoll.
Eftersom valet av kommunikationsprotokoll för en IoT-klient beror på
tillverkaren av klienten, måste IoT-systemen stöda ett flertal protokoll.
IoT-plattformarna kan normalisera protokollen genom att erbjuda verktyg eller
tjänster för att förvandla om de inkommande datapaketen till plattformarnas
egna format.

\subsection{Övriga drag}


% --------------------------------------------------------------------

\pagebreak
\section{Resultat}
\label{sec:esimluku}

\subsection{Axeda}
Axeda erbjuder sin plattform i form av en molntjänst. Plattformen omfattar
flera skikt, ända från grafiska användargränssnitt till själva hårdvaran.
Själva plattformen är proprietär programvara, men Axeda erbjuder flera 
gränssnitt för att få tillgång till plattformens funktioner. \cite{axeda}

En utvecklare kan integrera en maskin till plattformen med hjälp av plattformens
programbibliotek. Programbiblioteket normaliserar och enkrypterar
dataflödet mellan maskinerna och molnet i enhet med plattformens eget
protokoll. \cite{axeda}

Molntjänsten lagrar information om de uppkopplade maskinernas fysiska lägen och
tillstånd. En utvecklare kan använda plattformens användargränssnitt för att
definiera regler för hur alarm och händelser från maskinerna skall hanteras.
Användargränssnittet stöder hantering av uppkopplade maskiner på distans och
all information som samlas in av plattformen kan sökas upp och visualiseras
genom modifierbara dashboards. \cite{axeda}


\subsection{ThingWorx}
ThingWorx erbjuder en plattform med fokuset på att utveckla IoT enablerade
applikationer. Plattformen består av verktyg för att skapa användargränssnitt,
anpassade till användarnas behov, och för att bestämma regler för hur
data flöder mellan systemets maskiner. Användargränssnitten kan utvecklas
både genom drag och släpp-metoder och programmering.
Plattformen erbjuder funktioner för dataanalys och interaktiv kollaboration
mellan utvecklarna av IoT-tjänster.
ThingWorx erbjuder integrationsmöjligheter med Axedas egen plattform.


\subsection{Intel}

\subsection{RacoWireless}
RacoWireless Omega Management Suite (OMS) är en molnbaserad plattform som lägger 
tyngdpunkten på spårning och lagring av data från trådlöst uppkopplade maskiner.
Plattformen erbjuder flera mobilvänliga dashboards för analysering och 
visualisering av information. \cite{racowireless}

Plattformen stöder både SOAP- och REST-gränssnitt, vars funktioner kan
användas för att hantera uppkopplade maskiner. Med hjälp av gränssnitten kan 
utvecklare hantera maskinerna på distans och bevaka dataflödet till och från
dem. Övriga funktioner som OMS erbjuder är fakturering av mobildata samt
SMS-kommunikation med de uppkopplade maskinerna. \cite{racowireless}

RacoWireless Omega DevCloud är ett verktyg som kan användas för att
integrera utomstående applikationer och maskiner till RacoWireless egen 
IoT-plattform. Integrationen sker med Omega DevClouds REST-APIs. RacoWireless
erbjuder även en mobilapplikation som kan simulera en egentlig 
uppkopplad klient. \cite{racowireless}


\subsection{EVERYTHNG}

\subsection{Exosite}
\Kommentti{NEW 14.04.2015}


\subsection{AllJoyn}
AllJoyn är en plattform med öppen källkod som möjliggör att maskiner kan
upptäcka varann över ett lokalt nätverk. Fastän AllJoyn är huvudsakligen
avsedd för kommunikation av närbelägna maskiner, har noder utvecklats för att
låta maskiner kommunicera med utomstående molntjänster. \cite{alljoyn}

AllJoyns ramverk innehåller alla verktyg för att utveckla AllJoyn-enablerade
applikationer. Ramverket omfattar funktioner för upptäckning, säkerhet, 
händelser och sessioner. AllJoyn erbjuder inga dashboards för grafisk
granskning av de uppkopplade klienterna. Däremot kan varje uppkopplad maskin
förmedla enkla grafiska komponenter som applikationer kan visa för användaren.
Dessa komponenter innehåller information om den uppkopplade maskinen och
erbjuder ett gränssnitt för att kontrollera maskinen genom applikationen.
\cite{alljoyn}



\subsection{SiteWhere}
SiteWhere är en IoT-plattform med öppen källkod och fungerar både lokalt
och i molnet. Utvecklare kan nå data i plattformen direkt genom plattformens
REST-gränssnitt eller via ett webläsarbaserat användargränssnitt. Plattformen
stöder långtids datalagring, integration med utomstående molntjänster och


\subsection{Wind River}
Wind River Edge Management System är en molnbaserad plattform som omfattar
nivåer från de fysiska maskinerna till gränssnittet för applikationer.
\cite{windriver}



\subsection{Carriots}
Carriots är en PaaS som kan köras både i det offentliga molnet och på privata
servrar. Plattformen är skalbar och kan hantera upp till
hundratals miljoner uppkopplade maskiner. Maskinerna kommunicerar med
plattformen genom Carriots säkrade REST-API.\cite{carriots}

De huvudsakliga funktionerna i Carriots är datalagring, hantering av
regler för alarm och händelser, integration till andra molntjänster och
distanshantering av uppkopplade maskiner. Plattformen möjliggör även hantering
av flera samtida projekt. Däremot erbjuder plattformen inga 
dashboards för visualisering av data, men utomstående applikationer kan 
integreras i Carriots för detta ändamål.\cite{carriots}

% -------------------------------------------------------------------

\pagebreak
\section{Analys}
\label{sec:esimluku}

Detta kapitel analysera

\subsection{Kommunikation}
Det finns inget enhetligt sätt för hur IoT-plattformarna hanterar 
kommunikationen mellan de uppkopplade maskiner och plattformen. 
Somliga plattformar kopplar klienterna till systemet genom egen mjukvara som
installeras på klienterna. Mjukvaran sköter kommunikationen mellan klienterna
och plattformen och binder därmed klienterna tätt till plattformen. Andra 
plattformar erbjuder endast gränssnitt för kommunikation mellan maskinerna och 
plattformen. Ansvaret för hur klienterna tar kontakt med plattformen förs över
till klientuvecklaren, vilket innebär en lösare koppling mellan klienterna
och plattformen.
\newline
\newline
Axeda och AllJoyn är exempel på plattformar som sköter uppkopplingen med hjälp
av sina egna programbibliotek. Klienterna samverkar med plattformarna
genom att kalla på funktioner i plattformarnas programbibliotek. 
Programbiblioteken hanterar data i form av objekt istället för REST-meddelanden,
vilket döljer nätkommunikationen från utvecklaren och förenklar därmed
utvecklingsprocessen. Nackdelen med användningen av plattformspecifika
programbibliotek är att klienterna är tätt bundna till plattformen.
Detta gör det svårare att skapa enhetliga standarder för samverkan mellan
olika IoT-plattformar, eftersom ett plattformspecifikt programbibliotek fungerar
endast med en plattform. Ett annat problem med dessa programbibliotek är att
för att en maskin skall kunna kopplas upp till plattformen, måste
plattformutvecklaren se till att ett programbibliotek är tillgängligt för
denna maskin. Ifall plattformutvecklaren inte stöder en viss maskinmodell måste
utvecklaren av IoT-tjänster välja en annan modell och är därmed beroende av
plattformutvecklarens beslut för stödda maskiner.
\newline
\newline
ThingWorx, Carriots och Exosite hanterar kommunikationen mellan klienterna
och plattformen med HTTP- eller socket-baserade gränssnitt medan 
klientutvecklaren själv implementerar kommunikationslogiken för klienten.
Detta innebär att utvecklaren av IoT-tjänster kan fritt välja en maskinmodell
med nätverkskort för klienten, vilket inte alltid är möjligt då en plattform 
kräver användningen av ett specifikt programbibliotek.
\pagebreak

\subsection{Applikationer}
Medan de undersökta plattformarna hanterar säkerhet och skalbarhet på liknande
sätt, varierar deras applikationsnivåer kraftigare. En del plattformar
erbjuder möjligheten att på grafisk väg framställa användargränssnitt.
Andra plattformar saknar verktyg för visualisering av information och erbjuder
endast gränssnitt för att nå informationen i IoT-systemet. Dessa plattformar
överlåter ansvaret till utvecklaren av IoT-tjänsten för att skapa
användargränssnitt. Det finns även plattformar som kombinerar dessa två sätt
genoma att erbjuda färdiga grafiska komponenter som utvecklarna kan utvidga
eller konfigurera enligt behov.
\newline
\newline
Fördelarna med grafiskt skapade dashboards är att utvecklingen av
användargränssnitt försnabbas och utvecklarna sparar arbete genom att 
återanvända existerande grafiska komponenter. Med hjälp av drag-och-släpp-
editering av dashboards, kan personer utan programmeringskunskaper skapa egna 
vyer med grafer och andra interaktiva komponenter enligt behov. Detta
ger personer inom t.ex. företagets ledning möjlighet att få en översikt över
IoT-systemets tillstånd och historia och förminskar därmed rapporteringsarbetet.
Nackdelen med grafiskt skapade användargränssnitt är att utvecklaren är 
begränsad till de komponenter som plattformutvecklaren erbjuder. Axeda har löst
detta problem genom att låta utvecklarna själv utvidga komponenterna.
Utvecklarna kan använda sig av egen källkod för att skapa vyer, anpassade för
specifika ändamål.
\newline
\newline
Carriots och EVRYTHNG har inga egna grafiska komponenter för visualisering
av data men erbjuder istället REST-baserade gränssnitt för att låta utvecklaren
hämta information från IoT-systemet. Utvecklaren kan välja mellan att skapa
de grafiska elementen eller att integrera utomstående dashboard-tjänster till
plattformen. Detta ger utvecklaren största möjliga flexibilitet i valet av hur 
användargränssnittet skall se ut och fungera.
\newline
\newline

\subsection{Skalbarhet och säkerhet}
De undersökta plattformarna har i allmänhet tagit ställning till skalbarhet
och säkerhet på samma sätt. Flera av plattformarna är gjorda för att skalas upp
från enskilda servrar till omfattande molntjänster. 


\subsection{Utvecklingsverktyg}
För att underlätta utvecklingsprocessen erbjuder majoriteten av de undersökta 
plattformarna verktyg för att hjälpa utvecklarna. Verktygen kan stöda själva
programmeringen av IoT-applikationer eller hjälpa utvecklarna genom
att förse dem med information och utvecklingsförslag.
Axeda, Carriots och EVRYTHNG erbjuder webläsarbaserade utvecklingsverktyg för 
att skapa, redigera och radera IoT-objekt. Flera av de undersökta plattformarna 
har forum där användare kan vara i kontakt med både plattformutvecklarna
och andra användare. Forumen är effektiva kanaler för rådgivning och delning
av exempel från verkliga IoT-applikationer. Plattformutvecklarna kan även vara
i nära kontakt med användarna och få förslag på förbättring gällande
plattformarna. 




\subsection{Datalagring}

\pagebreak
\section{Sammanfattning}

\subsection{Företagsimplikationer}
Företag kan utnyttja resultatet i detta arbete för att göra exaktare beslut
vid anskaffning av IoT-plattformar. Eftersom arbetet identifierat de
huvudsakliga egenskaperna för de, i arbetet, nämnda plattformerna, kan företag 
försnabba beslutsprocessen genom att utesluta de plattformer som inte lämpar sig
för behovena.

\subsection{Förslag på fortsatt forskning}
Som fortsatt forskning föreslås en utvidgad empirisk undersökning genom att
öka antalet undersökta IoT-plattformar. 

Alternativt kunde ett antal av de tillgängliga IoT-plattformarna undersökas 
djupare, genom att granska egenskaperna hos plattformarna på en mer detaljerad
nivå. Detta kan utföras genom att undersöka både allmänt tillgänglig 
dokumentation av plattformarna och information som endast kan fås av 
plattformarnas ägare.

Ett annat förslag för vidare forskning är mätning av prestandan för egenskaperna 
hos plattformarna. Exempelvis kunde skalbarheten hos två eller flera plattformar
jämföras genom att undersöka hur dessa funktionerar under hög belastning.
Säkerheten hos plattformarna kunde även granskas med hjälp av
penetreringstester.

