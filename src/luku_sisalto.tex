% --------------------------------------------------------------------

\section{Inledning}
Den snabba utevecklingen av Internet of Things har lett till 

\subsection{Teoretisk bakgrund}
Info om IIC och deras planer

\subsection{Problembeskrivning}
Ett flertal IoT plattformer har nått användare under den senaste tiden.
Antalet tillämpningsområden för IoT enablerade system är massivt,
vilket gör det arbetsdrygt för utvecklaren att välja en lämplig plattform
till sitt behov.

\subsection{Problemformulering och syfte}
Syftet med arbetet är att kartlägga tillängliga IoT plattformer i förhållande
till varandra på basis av deras egenskaper. Samtidigt kommer arbetet
att identifiera gemensamma drag mellan plattformerna. 

\subsection{Avgränsningar}
Arbetet lägger fokuset på plattformernas funktionaliter, egenskaper, förmågor
och tillämpningsmiljöer.
Arbetet tar inte ställning till plattformernas ställning på marknaden.
Arbetet handlar inte heller om varje plattforms tekniska arkitektur.

\subsection{Material och metoder}
Forskningen utfördes i form av en empirisk undersökning. Information om varje
plattform hittades från plattformutvecklarnas nätsidor och dokumentation.



% --------------------------------------------------------------------

\section{Resultat}
\label{sec:esimluku}

\subsection{Plattformernas egenskaper}

\subsubsection{Axeda}

\subsubsection{SiteWhere}

\subsubsection{ThingWorx}

\subsection{Gemensamma drag mellan plattformerna}

% --------------------------------------------------------------------

\section{Analys}
\label{sec:esimluku}


\section{Sammanfattning}

