% --------------------------------------------------------------------

\section{Inledning}
Internet of Things (IoT) är ett av de mest aktuella ämnen i företagsverksamhet
och forskning. I industrin förses existerande maskiner med sensorer, som
skickar och mottar information över Internet. Fel kan upptäckas proaktivt
genom att analysera dataflödet från apparaterna och därmed minska
underhållskostnader.
På konsumentmarknaden utvecklas kroppsnära teknik, som samverkar med
mobiltelefoner och synkroniseras med molntjänster.

Ju större och mera komplexa IoT enablerade system blir, desto mera tid och
kapital krävs det för att utveckla och upprätthålla dem.
För att simplifiera och effektivera utvecklingen och underhållningen av
IoT system, har IoT plattformar utvecklats. Plattformarna kan erbjudas
som kompletta tjänster eller som programmeringsverktyg i form av
integrerade utvecklingsmiljöer (IDE).


\Kommentti{Lämna bort det här helt och hållet?}

\subsection{Problembeskrivning}
Eftersom IoT systemenas egenskaper och krav varierar från fall till fall,
gör även de plattformar som erbjuds för att utveckla dem. För att kunna
välja rätt plattform till rätt ändamål, krävs det information om hurdana
egenskaper och funktioner plattformar har.

Ett IoT system kan bestå av maskiner som inte ligger innanför samma
interna nätverk. De kan även bearbeta information, som endast utnämnda
personer får hantera. Då är det viktigt att plattformen stöder
funktioner för datasäkherhet och hantering av användarroller.
Ett annat system kan expandera snabbt på en kort tid genom att öka dess antal
IoT enablerade objekt. Då är det viktigt att plattformen erbjuder möjligheter
för horisontell skalbarhet.

Det ökande antalet IoT plattformar på marknaden gör valet arbetsdrygt.
Därför behövs det kompilerad information om hur plattformarna ställer sig
mot varandra.


\subsection{Problemformulering och syfte}
Syftet med arbetet är att kartlägga tillängliga IoT plattformar i förhållande
till varandra på basis av deras egenskaper. Samtidigt kommer arbetet
att identifiera gemensamma drag mellan plattformarna. 

\subsection{Avgränsningar}
Arbetet lägger fokuset på plattformarnas funktionaliter, egenskaper, förmågor
och tillämpningsmiljöer.
Arbetet tar inte ställning till plattformarnas ställning på marknaden.
Arbetet handlar inte heller om varje plattforms tekniska arkitektur.

\subsection{Material och metoder}
Forskningen utfördes i form av en empirisk undersökning. Information om varje
plattform hittades från plattformutvecklarnas nätsidor och dokumentation.

\pagebreak



% --------------------------------------------------------------------

\section{Internet of Things-plattform}
\label{sec:esimluku}


\subsection{Teknisk plattform}

En teknisk plattform är en kombination av hårdvara och mjukvara, som
tillsammans bildar en miljö, där applikationer kan utvecklas och köras i.
Plattformar underlättar utvecklingen av applikationer, eftersom utvecklaren
kan fokusera på själva lösningen istället och delegera infrastrukturen
till plattformen.
Detta gäller även i fallet av IoT plattformar. Utvecklaren kan allokera
största delen av sina resurser till att skapa värde för slutanvändaren,
medan plattformen ansvarar för de funktioner som stöder utvecklingen.

\subsection{Utmaningar för IoT-system}

\subsubsection{Skalbarhet}
Skalbarhet är ett mått på hur ett datasystem hanterar ett ökande antal 
uppkopplade maskiner \cite{scalability_def}. Ett horisontellt skalbart system
kan öka det maximala antalet maskiner som kan kopplas upp genom att öka
antalet fysiska servrar.
Exempelvis kan ett horisontellt skalbart system hantera 1000 uppkopplade
enheter med 1 server. För att hantera 2000 uppkopplade enheter, måste systemet
öka antalet servrar till 2. Förhållandet mellan antalet uppkopplade enheter
och antalet servrar förblir konstant, då antalet uppkopplade enheter ökar.

IoT-system måste klara av att hantera ett växande antal uppkopplade maskiner
på ett stabilt sätt. För att systemet som helhet skall vara skalbart, måste
även den underliggande plattformen vara det.

\subsubsection{Sammanslagning av data}
Information förmedlas i form av alarm och händelser mellan IoT-enablerade
maskiner och det centrala systemet. I systemet är dessa alarm och händelser
bundna till åtgärder som utförs då informationen når systemet. Åtgärderna
kan vara enkla, som en uppdatering \cite{mashup_def} 


\subsubsection{Säkerhet}

\subsubsection{Kommunikationsprotokoll}

\subsubsection{Big data}


% --------------------------------------------------------------------

\section{Metod}
\label{sec:esimluku}
\Kommentti{Skriv in kort backgrundsinformation om företagen}
Forskningen utfördes som en explorativ, kvalitativ undersökning och
Industrial Internet Consortiums (IIC) nätsida fungerade som startpunkt för den.
IIC är ett konsortium, bildat av bl.a. Intel, IBM och AT\&T, som jobbar
för att utveckla \Kommentti{standarder, och gemensamma arkitekturer} inom IoT.
Till IICs medlemmar hör över 100 företag, universitet och inrättningar, som
delar på ansvaret för att utveckla globala IoT lösningar.
Genom att granska medlemmarnas nätsidor, identifierades IoT plattformar.
Ifall en medlem samarbetade med ett eller flera företag, som också länkades
till IoT, kunde forskningen utvidgas till dessa företag.
Information om plattformarnas egenskaper och funktioner hittades genom
utvecklarnas nätsidor, vitböcker och plattformarnas dokumentation.

\Kommentti{Jämför resultatet med teorin -> kategoriserar plattformerna
enligt egenskaperna}

\Kommentti{Avvikande fall -> bilda egen grupp}

\Kommentti{Ta i beaktande trovärdigheten på vad företagen skriver. Om
möjligt så granskades dokumentationen för att verifiera att plattformerna
stöder de funktioner som de säger sig göra. Cross-checking mellan 
sidorna för att kolla partners}

\section{Resultat}
\label{sec:esimluku}

\subsection{Axeda}
Inledning:
Namn 

"Utplacering":
Open source/propertiär, cloud/on-premise

Hög nivå:
Vilka nivåer

Komponent nivå:
Kort om varje nivå

%%%%%%%
Axeda erbjuder sin plattform i form av en molntjänst. Plattformen omfattar
flera skikt, ända från grafiska användargränssnitt till själva hårdvaran.
Själva plattformen är proprietär programvara, men Axeda erbjuder flera 
gränssnitt för att få tillgång till plattformens funktioner.

En utvecklare kan integrera en maskin till plattformen med hjälp av plattformens
programbibliotek. Programbiblioteket normaliserar och enkrypterar
dataflödet mellan maskinerna och molnet i enhet med plattformens eget
protokoll.

Molntjänsten lagrar information om de uppkopplade maskinernas fysiska lägen och
tillstånd. En utvecklare kan använda plattformens användargränssnitt för att
definiera regler för hur alarm och händelser från maskinerna skall hanteras.
Användargränssnittet stöder hantering av uppkopplade maskiner på distans och
all information som samlas in av plattformen kan sökas upp och visualiseras
genom modifierbara dashboards.




%% Old stuff BEGIN
Axeda erbjuder sin plattform som en molntjänst. Plattformen stöder 
funktioner på flera skikt, ända från applikationsnivån till datapaketnivån.
På applikationsnivå erbjuder plattformen grafiska användargränssnitt för
visualisering av insamlad information från IoT systemet. Plattformen erbjuder
även grafiska verktyg för att hantera uppkopplade maskiner på distans.
För att kunna integrera maskiner med Axedas molntjänst, erbjuder plattformen 
verktyg för detta ändamål. Verktygena förser maskiner, som överför data 
trådlöst, med stöd för Axedas eget kommunikationsprotokoll AMMP 
(Adaptive Machine Messaging Protocol). Plattformen innehåller även verktyg
som hanterar användarrättigheter mellan maskinerna.
Axedas molntjänst är byggd för horisontell skalning. Tjänsten är godkänd
enligt ISO 27001 standaren, vilket innebär att infrastrukturen uppfyller
standardenliga säkerhetskrav. \Kommentti{Referens till standarden}
%% Old stuff END

\subsection{ThingWorx}
ThingWorx erbjuder en plattform med fokuset på att utveckla IoT enablerade
applikationer. Plattformen består av verktyg för att skapa användargränssnitt,
anpassade till användarnas behov, och för att bestämma regler för hur
data flöder mellan systemets maskiner. Användargränssnitten kan utvecklas
både genom drag och släpp-metoder och programmering.
Plattformen erbjuder funktioner för dataanalys och interaktiv kollaboration.
ThingWorx erbjuder integrationsmöjligheter med Axedas egen plattform.


\subsection{Intel}

\subsection{Raco Wireless}

\subsection{EVERYTHNG}

\subsection{Exosite}

\subsection{AllJoyn}

\subsection{SiteWhere}

\subsection{Windriver}

\subsection{Carriots}

% -------------------------------------------------------------------

\section{Analys}
\label{sec:esimluku}



\section{Sammanfattning}

